%%
%% This is file `sample-authordraft.tex',
%% generated with the docstrip utility.
%%
%% The original source files were:
%%
%% samples.dtx  (with options: `authordraft')
%% 
%% IMPORTANT NOTICE:
%% 
%% For the copyright see the source file.
%% 
%% Any modified versions of this file must be renamed
%% with new filenames distinct from sample-authordraft.tex.
%% 
%% For distribution of the original source see the terms
%% for copying and modification in the file samples.dtx.
%% 
%% This generated file may be distributed as long as the
%% original source files, as listed above, are part of the
%% same distribution. (The sources need not necessarily be
%% in the same archive or directory.)
%%
%% The first command in your LaTeX source must be the \documentclass command.
\documentclass[sigconf,  review=false, nonacm=true]{acmart}
%% NOTE that a single column version may required for 
%% submission and peer review. This can be done by changing
%% the \doucmentclass[...]{acmart} in this template to 
%% \documentclass[manuscript,screen]{acmart}
%% 
%% To ensure 100% compatibility, please check the white list of
%% approved LaTeX packages to be used with the Master Article Template at
%% https://www.acm.org/publications/taps/whitelist-of-latex-packages 
%% before creating your document. The white list page provides 
%% information on how to submit additional LaTeX packages for 
%% review and adoption.
%% Fonts used in the template cannot be substituted; margin 
%% adjustments are not allowed.

%%
%% \BibTeX command to typeset BibTeX logo in the docs
\AtBeginDocument{%
  \providecommand\BibTeX{{%
    \normalfont B\kern-0.5em{\scshape i\kern-0.25em b}\kern-0.8em\TeX}}}



%%
%% Submission ID.
%% Use this when submitting an article to a sponsored event. You'll
%% receive a unique submission ID from the organizers
%% of the event, and this ID should be used as the parameter to this command.
%%\acmSubmissionID{123-A56-BU3}

%%
%% The majority of ACM publications use numbered citations and
%% references.  The command \citestyle{authoryear} switches to the
%% "author year" style.
%%
%% If you are preparing content for an event
%% sponsored by ACM SIGGRAPH, you must use the "author year" style of
%% citations and references.
%% Uncommenting
%% the next command will enable that style.
%%\citestyle{acmauthoryear}

%%
%% end of the preamble, start of the body of the document source.
\usepackage{multirow}
\usepackage{float}
\begin{document}

%%
%% The "title" command has an optional parameter,
%% allowing the author to define a "short title" to be used in page headers.
\title{Detecting Sarcasm and Irony in Texts: An Overview}

%%
%% The "author" command and its associated commands are used to define
%% the authors and their affiliations.
%% Of note is the shared affiliation of the first two authors, and the
%% "authornote" and "authornotemark" commands
%% used to denote shared contribution to the research.
\author{Denis Reibel}
\email{denis.reibel@uni-ulm.de}
\affiliation{%
  \institution{Institute of Databases and Information Systems}
  \city{University Ulm}
  \country{Germany}
}

%%
%% By default, the full list of authors will be used in the page
%% headers. Often, this list is too long, and will overlap
%% other information printed in the page headers. This command allows
%% the author to define a more concise list
%% of authors' names for this purpose.
\renewcommand{\shortauthors}{Denis Reibel}

%%
%% The abstract is a short summary of the work to be presented in the
%% article.
\begin{abstract}
  As typical the abstract will be done very late, when most of the work is already written.
\end{abstract}

%%
%% Keywords. The author(s) should pick words that accurately describe
%% the work being presented. Separate the keywords with commas.
\keywords{natural language processing, sentiment analysis, sarcasm}

%% A "teaser" image appears between the author and affiliation
%% information and the body of the document, and typically spans the
%% page.
\begin{teaserfigure}
\vspace*{\fill} 
\begin{quote}
\centering 
\Large If sarcasm and irony are even for humans hard to recognize, how can we expect computers to do it?
\end{quote}
\vspace*{2 pt}
\end{teaserfigure}

%%
%% This command processes the author and affiliation and title
%% information and builds the first part of the formatted document.
\maketitle


\section{Introduction}

The first Chapter is there to give a gentle Introduction to the topic. In general I outlined where sarcasm and Irony do appear and why they're problematic.
Further I made clear why it's difficult to detect them. 
The questions I want to address in this work are:
\begin{enumerate} 
	\item What is the difference between irony and sarcasm and how do humans understand them?
	\item What are the main problems when detecting sarcasm or irony in texts?
	\item What good baseline algorithms for sarcasm detection exist? How do more complex perform in comparison to those baselines?
\end{enumerate}
In general I want to provide an overlook over the sarcasm-detection research field, that can also be understood by beginners in text analytics.


%Intro into Topic
%Where does sarcasm appear
%why is it difficult to deal with it?
%why is it important to deal with it
%rise of importance
%Structure of this work

\section{Language Theory}

Chapter Two is there to make clear what normally is understood as sarcasm and irony. I gave 2-3 definitions that are already existent ( i.e. Oxford dictionary, Cambridge Dictionary, Collins Dictionary), to give a better understanding what these concepts are. Also I tried to differentiate between those two concepts, but found out there is no clear border and an actual an scientific dispute whether they are the same or different concepts. \cite{Irony-as-relevant-inappropriateness} \cite{Psychological_aspects_of_irony_understanding}  \cite{diff_irony_sarcasm}
Further I shortly described how humans comprehend sarcasm.

%Definition of Sarcasm and Irony
%Distinction between those two
%How can one recognize sarcasm? How do humans recognize sarcasm?

\section{Problems in Sarcasm Detection}

This section is there to explain in more detail what the main Problems in Sarcasm Detection are and why it is so difficult to handle. The problems I wrote about are:
\begin{itemize} 
	\item Problems working with twitter / social media data \cite{The-effect-of-preprocessing-techniques-on-Twitter-sentiment-analysis} \cite{Analysis-of-Twitter-Specific-Preprocessing-Technique-for-Tweets}
	\item Fast paced change of language makes (i.e. slang words) are challenging the predictive models \cite{Sarcasm_detection_of_tweets}
    \item No gold standard for data annotation (even humans label sarcastic twets wrong)
	\item Imbalanced datasets (more non-sarcastic than sarcastic tweets)
    \item The role of context for sarcasm understanding and why computers struggle with context understanding \cite{The-Role-of-Conversation-Context-for-Sarcasm-Detection-in-Online-Interactions} \cite{Twitter-Sarcasm-Detection-Exploiting-a-Context-Based-Model}
\end{itemize}

%What are the main Problems?
%Why is so hard to deal with it
%Inherent problem with microblogging data

\section{Methods}

This is the main section, listing what other researchers already have achieved in this field. I grouped similar approaches into subsections (i.e. all neural network based attempts are in the subsection 'Deep Learning'). The subsections I've identified are:

\begin{enumerate} 
    \item Baseline Methods \cite{A-multidimensional-approach-for-detecting-irony-in-Twitter} \cite{Clues-for-Detecting-Irony-in-User-Generated-Contents}
    \item Pattern based Approaches \cite{pattern}
    \item Rule based Approaches \cite{Parsing-based-sarcasm-sentiment-recognition-in-Twitter-data}
    \item Learning based Methods \cite{baseline2}
    \item Deep Learning Approaches \cite{rnn}
\end{enumerate}

For each subsection I am describing the general Idea behind it. Also I selected 1-2 papers per section that are explained more detailed. (I'm not sure yet how detailed I want to describe these papers? I tend towards only describing what their key ideas are, because otherwise readers may be overwhelmed when being confronted with all those different models.) Those works also are compared in the following sections. 

%Present different methods that are used
%What baselines exist, that complex deep learning models can be compared with?
%To-Do: List of Methods
%Focus on x state of the art methods

\section{Comparison}

The central part of this section is a table, which contains the result achieved by the previous papers, in the following manner:

% Please add the following required packages to your document preamble:
% 
% Please add the following required packages to your document preamble:
\begin{table}[H]
\begin{tabular}{|l|cc|cc|c|}
\hline
\multirow{2}{*}{} & \multicolumn{2}{c|}{baseline}          & \multicolumn{2}{c|}{pattern-base}  & ... \\ \cline{2-6} 
                  & \multicolumn{1}{c|}{{[}1{]}} & {[}2{]} & \multicolumn{1}{c|}{{[}3{]}} & ... & ... \\ \hline
Accuracy          & \multicolumn{1}{c|}{x11}     & /       & \multicolumn{1}{c|}{x13}     & ... & ... \\ \hline
Precision         & \multicolumn{1}{c|}{x21}     & x22     & \multicolumn{1}{c|}{x23}     & ... & ... \\ \hline
Recall            & \multicolumn{1}{c|}{x31}     & /       & \multicolumn{1}{c|}{x33}     & ... & ... \\ \hline
F1-Score          & \multicolumn{1}{c|}{x41}     & /       & \multicolumn{1}{c|}{x43}     & ... & ... \\ \hline
\end{tabular}
\end{table}
Based on this table I plan to compare the papers and the method-groups. (I'm not sure whether it's a good idea to compare methods by their performance kpi's since they can differ i.e. depending on the used data. But I haven't found a better way to compare them.)

\section{Discussion}

In the Discussion section I will compare, the pro's and con's of the methods introduced in the previous chapters. Also explaining some potential weak-spots and solutions for those.
Maybe even other works have already addressed those weak-spots.

%Compare methods
%Discuss Pro's and Con's
\section{Excourse: Other Languages and sarcasm}
If I find enough space and time I plan to give an short excourse about the stand of sarcasm detection in languages other than english. I already found a few interesting works that I'd like to mention here. \cite{Sarcasm-Detection-on-Czech-and-English-Twitter} \cite{The-perfect-solution-for-detecting-sarcasm-in-tweets-not}

\section{Conclusion}

To finalize the work, this chapter summarizes the discussed topics and concludes the most important points. 
Also some ideas for future work will be presented.



%%
%% The next two lines define the bibliography style to be used, and
%% the bibliography file.
\bibliographystyle{ACM-Reference-Format}
\bibliography{sample-base}

%%
%% If your work has an appendix, this is the place to put it.
\appendix

\end{document}
\endinput
%%
%% End of file `sample-authordraft.tex'.
